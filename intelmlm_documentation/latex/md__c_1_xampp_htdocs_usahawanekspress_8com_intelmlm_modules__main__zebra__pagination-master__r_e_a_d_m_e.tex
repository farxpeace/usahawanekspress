\paragraph*{A generic pagination library that automatically generates navigation links}

A generic pagination script that automatically generates navigation links as well as next/previous page links, given the total number of records and the number of records to be shown per page. Useful for breaking large sets of data into smaller chunks, reducing network traffic and, at the same time, improving readability, aesthetics and usability.

Adheres to pagination best practices (provides large clickable areas, doesn't use underlines, the selected page is clearly highlighted, page links are spaced out, provides \char`\"{}previous page\char`\"{} and \char`\"{}next page\char`\"{} links, provides \char`\"{}first page\char`\"{} and \char`\"{}last page\char`\"{} links -\/ as outlined in an article by Faruk Ates from 2007, which can now be found \href{https://gist.github.com/622561}{\tt here}, can generate links both in natural as well as in reverse order, can be easily, localized, supports different positions for next/previous page buttons, supports page propagation via G\-E\-T or via U\-R\-L rewriting, is S\-E\-O-\/friendly, and the appearance is easily customizable through C\-S\-S.

Please note that this is a {\itshape generic} pagination script, meaning that it does not display any records and it does not have any dependencies on database connections or S\-Q\-L queries, making it very flexible! It is up to the developer to fetch the actual data and display it based on the information returned by this pagination script. The advantage is that it can be used to paginate over records coming from any source like arrays or databases.

The code is heavily commented and generates no warnings/errors/notices when P\-H\-P's error reporting level is set to E\-\_\-\-A\-L\-L.

\subsection*{Features}


\begin{DoxyItemize}
\item it is a generic library\-: can be used to paginate records both from an array or from a database
\item it automatically generates navigation links, given the total number of items and the number of items per page (examples of best practices are also included)
\item navigation links can be generated in natural or in reverse order
\item is S\-E\-O-\/friendly – it uses rel=”next” and rel=”prev” and solves the problem of duplicate content on the first page without navigation and the first page having the page number in the U\-R\-L
\item appearance is easily customizable through C\-S\-S
\item code is heavily commented and generates no warnings/errors/notices when P\-H\-P’s error reporting level is set to E\-\_\-\-A\-L\-L
\item has comprehensive documentation
\end{DoxyItemize}

\subsection*{Requirements}

P\-H\-P 5+

\subsection*{How to use}

Make sure that in the $<$head$>$ of your page you have

```html $<$link rel=\char`\"{}stylesheet\char`\"{} href=\char`\"{}path/to/zebra\-\_\-pagination.\-css\char`\"{} type=\char`\"{}text/css\char`\"{}$>$ ```

If you want to preserve hashes in the U\-R\-L, also include the Java\-Script file – simply including it will suffice; (j\-Query needs to also be loaded before loading this file)

```javascript $<$script type=\char`\"{}text/javascript\char`\"{} src=\char`\"{}path/to/zebra\-\_\-pagination.\-js\char`\"{}$>$$<$/script$>$ ```

Paginate data from an array\-:

```php $<$?php // let's paginate data from an array... \$countries = array( // array of countries );

// how many records should be displayed on a page? \$records\-\_\-per\-\_\-page = 10;

// include the pagination class require 'path/to/\-Zebra\-\_\-\-Pagination.\-php';

// instantiate the pagination object \$pagination = new \hyperlink{class_zebra___pagination}{Zebra\-\_\-\-Pagination()};

// the number of total records is the number of records in the array \$pagination-\/$>$records(count(\$countries));

// records per page \$pagination-\/$>$records\-\_\-per\-\_\-page(\$records\-\_\-per\-\_\-page);

// here's the magick\-: we need to display {\itshape only} the records for the current page \$countries = array\-\_\-slice( \$countries, ((\$pagination-\/$>$get\-\_\-page() -\/ 1) $\ast$ \$records\-\_\-per\-\_\-page), \$records\-\_\-per\-\_\-page );

?$>$

\begin{TabularC}{0}
\hline
\end{TabularC}


Country

$<$?php foreach (\$countries as \$index =$>$ \$country)\-:?$>$

$<$tr$<$?php echo \$index \% 2 ? ' class=\char`\"{}even\char`\"{}' \-: '')?$>$$>$ 

$<$?php echo \$country?$>$ 

$<$?php endforeach?$>$ 

$<$?php

// render the pagination links \$pagination-\/$>$render();

?$>$ ```

Paginate data from My\-S\-Q\-L\-:

```php $<$?php // how many records should be displayed on a page? \$records\-\_\-per\-\_\-page = 10;

// include the pagination class require 'path/to/\-Zebra\-\_\-\-Pagination.\-php';

// instantiate the pagination object \$pagination = new \hyperlink{class_zebra___pagination}{Zebra\-\_\-\-Pagination()};

// the My\-S\-Q\-L statement to fetch the rows // note how we build the L\-I\-M\-I\-T // also, note the \char`\"{}\-S\-Q\-L\-\_\-\-C\-A\-L\-C\-\_\-\-F\-O\-U\-N\-D\-\_\-\-R\-O\-W\-S\char`\"{} // this is to get the number of rows that would've been returned if there was no L\-I\-M\-I\-T // see \href{http://dev.mysql.com/doc/refman/5.0/en/information-functions.html#function_found-rows}{\tt http\-://dev.\-mysql.\-com/doc/refman/5.\-0/en/information-\/functions.\-html\#function\-\_\-found-\/rows} \$\-My\-S\-Q\-L = ' S\-E\-L\-E\-C\-T S\-Q\-L\-\_\-\-C\-A\-L\-C\-\_\-\-F\-O\-U\-N\-D\-\_\-\-R\-O\-W\-S country F\-R\-O\-M countries L\-I\-M\-I\-T ' . ((\$pagination-\/$>$get\-\_\-page() -\/ 1) $\ast$ \$records\-\_\-per\-\_\-page) . ', ' . \$records\-\_\-per\-\_\-page . ' ';

// if query could not be executed if (!(\$result = (\$\-My\-S\-Q\-L))) \{ \begin{DoxyVerb}// stop execution and display error message
die(mysql_error());
\end{DoxyVerb}


\}

// fetch the total number of records in the table \$rows = mysql\-\_\-fetch\-\_\-assoc(mysql\-\_\-query('S\-E\-L\-E\-C\-T F\-O\-U\-N\-D\-\_\-\-R\-O\-W\-S() A\-S rows'));

// pass the total number of records to the pagination class \$pagination-\/$>$records(\$rows\mbox{[}'rows'\mbox{]});

// records per page \$pagination-\/$>$records\-\_\-per\-\_\-page(\$records\-\_\-per\-\_\-page);

?$>$

\begin{TabularC}{0}
\hline
\end{TabularC}


Country

$<$?php \$index = 0?$>$

$<$?php while (\$row = mysql\-\_\-fetch\-\_\-assoc(\$result))\-:?$>$

$<$tr$<$?php echo \$index++ \% 2 ? ' class=\char`\"{}even\char`\"{}' \-: ''?$>$$>$ 

$<$?php echo \$row\mbox{[}'country'\mbox{]}?$>$ 

$<$?php endwhile?$>$ 

$<$?php

// render the pagination links \$pagination-\/$>$render();

?$>$ ```

Paginate data from My\-S\-Q\-L in reverse order\-:

```php $<$?php // how many records should be displayed on a page? \$records\-\_\-per\-\_\-page = 10;

// include the pagination class require 'path/to/\-Zebra\-\_\-\-Pagination.\-php';

// instantiate the pagination object \$pagination = new \hyperlink{class_zebra___pagination}{Zebra\-\_\-\-Pagination()};

// show records in reverse order \$pagination-\/$>$reverse(true);

// when showing records in reverse order, we need to know the total number // of records from the beginning if (!(\$result = ('S\-E\-L\-E\-C\-T C\-O\-U\-N\-T(id) A\-S records F\-R\-O\-M countries'))) \begin{DoxyVerb}die (mysql_error());
\end{DoxyVerb}


// pass the total number of records to the pagination class \$pagination-\/$>$records(array\-\_\-pop(mysql\-\_\-fetch\-\_\-assoc(\$result)));

// records per page \$pagination-\/$>$records\-\_\-per\-\_\-page(\$records\-\_\-per\-\_\-page);

// the My\-S\-Q\-L statement to fetch the rows // note the L\-I\-M\-I\-T -\/ use it exactly like that! // also note that we're ordering data descendingly -\/ most important when we're // showing records in reverse order! \$\-My\-S\-Q\-L = ' S\-E\-L\-E\-C\-T country F\-R\-O\-M countries O\-R\-D\-E\-R B\-Y country D\-E\-S\-C L\-I\-M\-I\-T ' . ((\$pagination-\/$>$get\-\_\-pages() -\/ \$pagination-\/$>$get\-\_\-page()) $\ast$ \$records\-\_\-per\-\_\-page) . ', ' . \$records\-\_\-per\-\_\-page . ' ';

// if query could not be executed if (!(\$result = (\$\-My\-S\-Q\-L))) \begin{DoxyVerb}// stop execution and display error message
die(mysql_error());
\end{DoxyVerb}


?$>$

\begin{TabularC}{0}
\hline
\end{TabularC}


Country

$<$?php \$index = 0?$>$

$<$?php while (\$row = mysql\-\_\-fetch\-\_\-assoc(\$result))\-:?$>$

$<$tr$<$?php echo \$index++ \% 2 ? ' class=\char`\"{}even\char`\"{}' \-: ''?$>$$>$ 

$<$?php echo \$row\mbox{[}'country'\mbox{]}?$>$ 

$<$?php endwhile?$>$ 

$<$?php

// render the pagination links \$pagination-\/$>$render();

?$>$ ```

Visit the $\ast$$\ast$\href{http://stefangabos.ro/php-libraries/zebra-pagination/}{\tt project's homepage}$\ast$$\ast$ for more information. 